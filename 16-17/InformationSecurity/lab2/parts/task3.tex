\section*{Сценарий 3}

\subsection*{Задание}
Злоумышленник опубликовал персональную информацию жертвы, включая имя,
фотографии, номер мобильного телефона и электронную почту на сайте знакомств
«для взрослых». В результате чего, жертва получает большое число нежелательных
звонков и сообщений.

\subsection*{Тип преступления}
\begin{itemize}
  \item Кража личности/идентичности.
  \item Публикация личных данных.
\end{itemize}

\subsection*{Способ совершения преступления}
\begin{itemize}
  \item Социальная инженерия.
  \item Сбор личных данных из открытых и закрытых источников (возможно кража
    закрытой компьютерной информации соответствующими способами).
\end{itemize}

\subsection*{Мотив}
\begin{itemize}
  \item Месть.
  \item Забава.
  \item Самоудовлетворение.
\end{itemize}

\subsection*{Круг подозреваемых}
Любой, кто мог узнать информации о потерпевшем. Люди, которые могут быть
заинтересованы в мести.

\subsection*{Применяемые статьи УК}
\begin{itemize}
  \item Статья 13.11 КоАП. Нарушение установленного законом порядка сбора,
    хранения, использования или распространения информации о гражданах
    (персональных данных)
  \item УК РФ. Нарушение неприкосновенности частной жизни
  \item 138 УК РФ (при похищении информации из закрытых источников).  Нарушение
    тайны переписки, телефонных переговоров, почтовых, телеграфных или иных
    сообщений
  \item 272 УК РФ (при похищении информации из закрытых ис
\end{itemize}

\subsection*{Лица несущие ответственность}
Злоумышленник / группа злоумышленников.

\subsection*{Литература} УК РФ.
