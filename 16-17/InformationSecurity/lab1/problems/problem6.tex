\begin{problem}
  Зашифровать фразу-палиндром, используя шифр перестановок.

  Палиндром: ''\textit{АРГЕНТИНА МАНИТ НЕГРА}''

  \begin{table}[H]
    \centering
    \begin{tabular}{|l|l|l|l|l|l|l|}
      \hline
      1                       & 2 & 3 & 4 & 5 & 6 & 7 \\ \hline
      \multicolumn{1}{|c|}{7} & 6 & 1 & 3 & 2 & 5 & 4 \\ \hline
    \end{tabular}
  \end{table}

\end{problem}

\begin{solution}
  Решение получено автоматически:
  \begin{minted}
  [
  frame=lines,
  framesep=2mm,
  baselinestretch=1.2,
  fontsize=\footnotesize,
  linenos
  ] 
  {python}
#!/bin/python3

d = [7, 6, 1, 3, 2, 5, 4]

s = "АРГЕНТИНАМАНИТНЕГРАДО"
result = ['*'] * len(s)

for i, c in enumerate(s):
    result[(d[i % 7] - 1) + 7 * (i // 7)] = c

print(''.join(result))
  \end{minted}

  Зашифрованный текст:
  \begin{addmargin}[2cm]{1cm}
    ГНЕИТРАМНАТИАНГАРОДЕН
  \end{addmargin}
\end{solution}
