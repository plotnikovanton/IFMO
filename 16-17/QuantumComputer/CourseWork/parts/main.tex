Фотоны являются нейтральными частицами, которые слабо взаимодействуют друг с
другом. Их можно практически без потерь переносить на большие расстояния по
оптическому волокну, задерживать при помощи фазовращателей и создавать
суперпозиции их состояний при помощи светоделителей.

\subsection{Физическая аппаратура}

Рассмотрим систему из квантовых резонаторов, в которой имеется двукратно,
вырожденная фотонная мода с частотой $\hbar\omega$. Таким образом два базисных
состояния кубита соответствуют одному фотону, находящемуся в первом резонаторе
($\ket{01}$), и одному фотону во втором ($\ket{10}$). Состояние кубита таким
образом можно представить в виде $c_0\ket{01} + c_1\ket{10}$, Такое состояние
будем называть двойственным. Будем рассматривать одиночные фотоны как волновые
пакеты, движущиеся в свободном пространстве, а не в резонаторе. Роль
резонаторов сводится к формированию двух различных пространственных мод.

Один из экспериментальных способов генерации одиночных фотонов состоит в
ослаблении излучения, испускаемого лазером. Лазер излучает когерентное
состояние $\ket{\alpha}$, определенное, как
\begin{equation}
  \ket{\alpha} = e^{\frac{|\alpha|^2}{2}} 
  \sum_{n=0}^\infty \frac{\alpha^n}{\sqrt{n!}}\ket{n},
\end{equation}
где $\ket{n}$ --- состояние, в котором резонансная мода заселена $n$ фотонами.

Например, при $\alpha = \sqrt{0.1}$ имеем состояние $\sqrt{0.9}\ket{0} +
\sqrt{0.09}\ket{1} + \sqrt{0.002}\ket{2} + \dots$. Таким образом, на выходе
аттенюатора будет получен один фотон с вероятностью больше $95\%$, а
вероятность ошибки составит $5\%$. Заметим, что в $90\%$ случаев через
аттенюатор не пройдет ни один фотон. Это значит, что полученный источник имеет
интенсивность всего $0,1$ фотона в единицу времени, кроме того, мы не можем
определить, когда фотон был испущен, а когда нет, поэтому нельзя
синхронизировать два таких источника.

Синхронизации можно достичь с помощью параметрического понижения частоты. Для
этого можно пропустить фотоны с частотой $\omega_0$ через нелинейный оптический
кристалл, ориентированный строго определенным образом, например $KH_2PO_4$, что
приведет к генерации фотонных пар на резонансных частотах $\omega_1 + \omega_2
= \omega0$. Если детектор зарегистрировал одиночный фотон $\omega_2$, то мы
узнаем о существовании одиночного фотона $\omega_1$. Установив на выходе
дополнительный селектор, открывающийся, только в том случае, когда детектор
зафиксировал один фотон, а не два или более, мы получим источник одиночных
фотонов. Можно синхронизировать во времени несколько таких источников, подбирая
правильную задержку во времени для каждой моды. Точность синхронизации
определяется способностью детектора и селектора.

\begin{figure}[H]
  \centering
  \begin{tikzpicture}
    \draw (0, 1) rectangle (2, 2) node[pos=.5] {Лазер};
    \draw [->] (2, 1.5) -- (4, 1.5) node[pos=.5, below] {$\omega_0$};
    \draw (4, 0.7) rectangle (6, 2.3) node[pos=.5] {Кристалл};
    \draw [->] (6, 1.5) -- (7.3, 2) node[pos=.5, above] {$\omega_1$};
    \draw [->] (6, 1.5) -- (7.3, 1) node[pos=.5, below] {$\omega_2$};
  \end{tikzpicture}
  \caption{Схема генерации одиночных фотонов методом параметрического понижения
  частоты}
\end{figure}

Существует много способов детектирования одиночных фотонов. Для нас важна
способность детектора с большой вероятностью правильно определять, имеется ли
на данной пространственной моде. На практике несовершенство детектора уменьшает
вероятность правильной регистрации одиночного фотона. Вероятность того, что
одиночный фотон, падающий на детектор, создаст фотопару, которая даст фклад в
фототок, называется квантовой эффективностью и обозначается как $\eta$ 
($0 \leq \eta \leq 1$). Детектор также характеризуется своей полосой
пропускания (временем отклика), уровнем шума и <<темновом>> током.

Экспериментальная техника для управления состояниями фотонов включает три
важных компонента: зеркала, фазовращатели и светоделители. Зеркала с большим
коэффициентом отражения отражают фотоны и меняют направление их распространения
в пространстве.

Фазовращатель представляет собой обычную прозрачную пластинку,
коэффициент преломления $n$ которой отличается от коэффициента преломления в
вакууме $n_0$.  Например, показатель преломления обычного боро-селикатного
стекла в оптическом диапазоне $n \approx 1.5n_0$. При прохождении фотона через
пластинку длины $L$ его фаза изменится на $e^{ikL}$, где
$k=\frac{n\omega}{c_0}$.

Светоделитель представляет собой частично посеребренное стекло с коэффициентом
отражения $R$ и коэффициентом порпускания $1-R$. Обычно он изготавливается из
двух призм и тонкого металлического слоя между ними.

Из нелинейной оптики известно что, показатель преломления $n$ некоторых веществ
зависит от полной интенсивности излучения $I$ по закону:
\begin{equation}
  n(I) = n + n_2I.
\end{equation}

Это явление известно как оптический эффект Керра. Представим себе эксперимент,
когда через такую среду проходят одновременно два луча света одинаковой
интенсивности, а их пути почти совпадают, тогда в случае прохождения этих двух
лучей каждый из них приобретает дополнительную фазу $e^{in_2IL\omega/c_0}$.
Это было бы хорошо, если бы длину пути $L$ можно было сделать сколь угодно
большой, но к сожалению, это невозможно, поскольку среды обладающие эффектом
Керра, сильно поглощают свет и излучение рассеивается в другие пространственные
моды.

\subsection{Квантовые вычисления}

Если квантовая информация закодирована в описанном выше представлении
$c0\ket{01} + c1\ket{10}$, то, использую фазовращатели, светоделители и среду
Керра, можно реализовать произвольное унитарное преобразование.

\subsubsection{Фазовращатель}

Фазовращатель замедляет моду излучения, проходящую через него. Это связано с
уменьшением скорости света в среде, имеющей больший показатель преломления, а
именно время за которое свет пройдет длину $L$ в среде с показателем
преломления $n$ чем в вакууме на $\Delta \equiv \frac{(n - n_0)L}{c_0}$.
Оператор, описывающий прохождение света через фазовращатель, обозначим как $P$.
Тогда действие $P$ на вакуумное состояние: $P\ket{0} = \ket{0}$, тогда как
для однофотонного состояния $P\ket{1} = e^{i\Delta}\ket{1}$.

В двойственном представлении $P$ осуществляет полезную логическую операцию.
Расположив фазовращатель на пути одной из двух мод, мы задержим вращение ее
фазы по отношению ко второй моде, прошедшей то же самое расстояние в вакууме.
Таким образом состояние $c0\ket{01} + c1\ket{10}$ перейдет в
$c_0e^{-i\Delta/2}\ket{01} + c_1e^{i\Delta/2}\ket{10}$. Заметим, что если
представить логические состояния как $\ket0_L=\ket{01}$ и $\ket1_L=\ket{10}$,
то эта операция представляет собой вращение вокруг оси $\hat{z}$, т.е.
\begin{equation}
  R_z(\Delta) = e^{\frac{-iZ\Delta}{2}},
\end{equation}
где Z --- матрица Паули $\sigma_Z$. Поэтому можно считать, что $P$ отисывает
эволюцию гамильтониана в течении времени $L/c_0$, т.е. $P=\exp(-iHL/c_0)$.

\subsection{Светиделитель}

Начме описание эволюционных состояний с гамильтониана. Напомним, что
светоделитель действует на две фотонные моды, которые мы будем описывать при
помощи операторов рождения $a^\dag$, $b^\dag$ и уничтожения $a$, $b$.
Гамильтониан имеет вид:
\begin{equation}
  H_{bs} = i \theta\left(ab^\dag - a^\dag b\right),
\end{equation}
а унитарный оператор, описывающий прохождение ререз светоделитель,
\begin{equation}
  B=\exp\left[\theta\left(a^\dag b - ab^\dag\right)\right].
\end{equation}

Рассмотрим операторы сопряженного действия оператора $B$ на операторы $a$ и
$b$ при помощи формулы Бейкера-Кэмпбелла-Хаустдорфа
$\left( e^{\lambda G}Ae^{-\lambda G} =
\sum_{n=0}^{\infty}\frac{\lambda^n}{n!}C_n \right)$.

\begin{equation}
  BaB^\dag = e^{\theta G}ae^{-\theta G}
  = \sum_{n=0}^{\infty} \frac{\theta^n}{n!} C_n
  = \sum_{n \text{четн}} \frac{{(i\theta)}^n}{n!} a
  + i \sum_{n \text{нечет}} \frac{{(i\theta)}^n}{n!} b
  = a \cos \theta - b \sin \theta.
\end{equation}

Аналогично для $b$, если поменять местами $a$ и $b$:
\begin{equation}
  BbB^\dag = -a\sin\theta + b\cos\theta.
\end{equation}

Прежде всего отметим, что $B \ket{00} = \ket{00}$, т.е. если ни на одной
входной ноде фотонов нет, то их не будет и на выходных нодах. Если на входе
емеется один фотон на моде $a$, то используя $\ket{1} = a^\dag\ket{0}$, находим
\begin{equation}
  B\ket{01} = Ba^\dag\ket{00} = Ba^\dag B^\dag B\ket{00}
  = \left( a^\dag \cos\theta + b^\dag \sin\theta \right) \ket{00}
  = \cos \theta\ket{01} + \sin \theta\ket{10}.
\end{equation}
Аначогично
\begin{equation}
  B\ket{10} = \cos\theta\ket{10} - \sin\theta\ket{01}.
\end{equation}
Таким образом, в базисе $\ket{0_L}$, $\ket{1_L}$ можно записать $B$ как
\begin{equation}
  B = \begin{bmatrix}
    \cos\theta & -\sin\theta \\
    \sin\theta & \cos\theta
  \end{bmatrix} = e^{i\theta Y}
\end{equation}

Используя фазовращатели и светоделители можно подействовать на кубит любым
унитарным оператором.

\subsubsection{Нелинейные керровские среды}

Квантовый гамильтониан, описывающий эффект Керра, имеет вид:
\begin{equation}
  H_{xpm} = -\chi a^\dag a b^\dag b,
\end{equation}
Соответственно, при прохождении сквозь кристал длины $L$ квантовое состояние
преобразуется унитарным оператором
\begin{equation}
  K = e^{i\chi La^\dag a b^\dag b}
\end{equation}

При помощи среды Керра и светоделителей можно реализовать операцию
\verb|CNOT|.
\begin{gather}
  K\ket{00} = \ket{00}, \\
  K\ket{01} = \ket{01}, \\
  K\ket{10} = \ket{10}, \\
  K\ket{11} = i^{i\chi L}\ket{11}.
\end{gather}
Выбор $\chi^L=\pi$ дает $K\ket{11}=-\ket{11}$. Два логических кубита в
двойственном представлении задаются четырьмя базисными состояниями
$\ket{e_{00}} = \ket{1001}$, $\ket{e_{01}} = \ket{1010}$,
$\ket{e_{10}} = \ket{0101}$, $\ket{e_{11}} = \ket{0110}$. Допустим, что мы
пропускаем вторую и третью моды через среду Керра. В этом случае
$K\ket{e_i} = \ket{e_i}$ для всех $i$ за исключением
$K\ket{e_{11}} = -\ket{e_{11}}$. Это почти то, что нужно, поскольку операция
\verb|CNOT| может быть записана как
\begin{equation}
  \underbrace{\begin{bmatrix}
    1 & 0 & 0 & 0 \\
    0 & 1 & 0 & 0 \\
    0 & 0 & 0 & 1 \\
    0 & 0 & 1 & 0
  \end{bmatrix}}_{U_{CN}}
  = \underbrace{\frac{1}{\sqrt2}
  \begin{bmatrix}
    1 & 1 & 0 & 0 \\
    1 & -1 & 0 & 0 \\
    0 & 0 & 1 & 1 \\
    0 & 0 & 1 & -1
\end{bmatrix}}_{I \otimes H}
  \times
  \underbrace{\begin{bmatrix}
    1 & 0 & 0 & 0 \\
    0 & 1 & 0 & 0 \\
    0 & 0 & 1 & 0 \\
    0 & 0 & 0 & -1
  \end{bmatrix}}_{K}
  \underbrace{\frac{1}{\sqrt2}
  \begin{bmatrix}
    1 & 1 & 0 & 0 \\
    1 & -1 & 0 & 0 \\
    0 & 0 & 1 & 1 \\
    0 & 0 & 1 & -1
  \end{bmatrix}}_{I \otimes H},
\end{equation}
где $H$ --- однокубитовый элемент Адмара (который реализуется с помощью
фазовращателей), а $K$ --- преобразование Керра с $\chi L = \pi$.



