\section{Введение}

При решении математических задач процесс разыскания решения поставленной
проблемы начинается с составления гипотетического плана ее решения, разбивки ее
на несколько частных вопросов, решение которых, по нашему расчету, приводит нас
к решению интересующей нас проблемы. Так, при решении геометрической задачи на
определение какой-либо геометрической величины через другие, мы рассчитываем
прийти к определению неизвестного через последовательное определение других
неизвестных. Приступая к решению перво­ го вопроса, а затем, в случае удачи,
второго и всех остальных вопросов, мы первым делом прибегаем к памяти, стараясь
подвести его, как частный случай, под уже известные нам проблемы.

Аналогия является, пожалуй одним из самых распространенных методов научного
исследования. Широкое применение аналогий часто приводит исследователя к более
или менее правдоподобным предположениям о свойствах изучаемого объекта, которые
могут быть затем подтверждены или опровергнуты опытом или более строгими
рассуждениями.

\subsection{Сущность аналогии и её виды}
Как метод исследования традукция заключается в том, что, установив сходство
двух объектов в некотором отношении, делают вывод о сходстве тех же объектов и
в другом отношении.

\begin{table}[H]
  \centering
  \begin{tabular}{c|cccc}
    Объекты & \multicolumn{4}{c}{Свойства объектов} \\
    \hline
    A & a & b & c & d$\dots$ \\
    B & a & b & c & x$\dots$ \\
  \end{tabular}
  \caption{Схема представления объектов}\label{tab:obj}
\end{table}

Методом аналогий из объектов представленнып таблицой~\ref{tab:obj} можно
сделать вывод или предположить, что свойство $d = x$.

Аналогии можно радлелить на типы:
\begin{enumerate}
  \item Простую аналогию, при которой по сходству объектов в некоторых
    признаках заключают их сходство в других признаках;

  \item Распространенную аналогию, при которой из сходства явлений делают вывод о
    сходстве причин.
\end{enumerate}
Простая аналогия в свою очередь может быть:
\begin{enumerate}
  \item строгой аналогией, при которой признаки сравниваемых объектов находятся
    во взаимной зависимости;
  \item нестрогой аналогией, при которой признаки сравниваемых объектов не
    находятся в явной взаимной зависимости.
\end{enumerate}
