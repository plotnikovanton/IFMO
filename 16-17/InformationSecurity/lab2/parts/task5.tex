\section*{Сценарий 5}

\subsection*{Задание}
Злоумышленником (внешним или сотрудником компании) были похищены исходные коды
ПО, позже проданные конкурирующей компании.


\subsection*{Тип преступления}
\begin{itemize}
  \item Мошенничество.
  \item Нарушение коммерческой тайны.
  \item Незаконный доступ к охраняемой информации.
\end{itemize}

\subsection*{Способ совершения преступления}
\begin{itemize}
  \item Социальная инженерия.
  \item Хищение.
  \item Прочие способы, предоставляющие доступ к аккуантам сотрудников.
\end{itemize}

\subsection*{Мотив}
\begin{itemize}
  \item Деньги.
  \item Самоудовлетворение.
  \item Месть.
\end{itemize}

\subsection*{Круг подозреваемых}
Сотрудники компании или недавно уволенные сотрудники. Не принятые на работу 
личности. Конкурентные компании. Любой человек, желающий обогатиться.

\subsection*{Применяемые статьи УК}
\begin{itemize}
  \item УК РФ, Статья 159. Мошенничество.
  \item УК РФ, Статья 183. Незаконные получение и разглашение сведений,
    составляющих коммерческую, налоговую или банковскую тайну.
  \item 272 УК РФ. Неправомерный доступ к компьютерной информации.
\end{itemize}

\subsection*{Лица несущие ответственность}
Злоумышленник / группа злоумышленников.

\subsection*{Литература} Федеральный закон ''О коммерческой тайне''. УК РФ.
