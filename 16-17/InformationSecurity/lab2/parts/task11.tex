\section*{Сценарий 11}

\subsection*{Задание}
Злоумышленниками была создана страница (в соц. сети, на сайте), содержащая
информацию, порочащую репутацию вашей компании/товара.

\subsection*{Тип преступления}
\begin{itemize}
  \item Незаконный доступ к охраняемой информации.
\end{itemize}

\subsection*{Способ совершения преступления}
\begin{itemize}
  \item Рассмотренные ранее способы получения доступа к защищенной информации.
  \item Создание страницы, размещение информации негативной направленности.
\end{itemize}

\subsection*{Мотив}
\begin{itemize}
  \item Месть.
  \item Самоудовлетворение.
  \item Конкурентное соперничество.
\end{itemize}

\subsection*{Круг подозреваемых}
\begin{itemize}
  \item Владелец страницы.
  \item Недовольные сотрудники компании.
  \item Бывшие сотрудники.
  \item Конкуренты компании.
\end{itemize}

\subsection*{Применяемые статьи УК}
\begin{itemize}
  \item 272 УК РФ. Неправомерный доступ к компьютерной информации.
  \item УК РФ, Статья 183. Незаконные получение и разглашение сведений,
    составляющих коммерческую, налоговую или банковскую тайну.
\end{itemize}

\subsection*{Лица несущие ответственность}
Лицо, разместившее информацию.

\subsection*{Литература} Федеральный закон ''О коммерческой тайне''. УК РФ.
