Представление кубита при помощи одиночных фотонов весьма привлекательно. Их
легко генерировать и детектировать. В двойственном представлении можно
реализовать любой произвольный однокубитовый оператор. К сожалению между
фотонами сложно обеспечить взаимодействие --- в самых лучших керровских средах
взаимодействие слишком слабое, чтобы получить перекрестную фазовую модуляцию
порядка $\pi$ между однофотонными состояниями. Кроме того эффект Керра
неизбежно сопровождается поглащением света. Т.о. построение квантового
компьютера на оптических фотонах имеет очень мало шансов на успех.
