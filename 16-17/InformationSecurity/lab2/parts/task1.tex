\section*{Сценарий 1}

\subsection*{Задание}

Электронная почта или аккаунт в социальной сети жертвы был взломан, после
жертва получила требования о выплате «выкупа» за возврат доступа к электронной
почте, аккаунту.

\subsection*{Тип преступления}
\begin{itemize}
  \item Кража компьютерной информации
  \item Вымогательство
\end{itemize}

\subsection*{Способ совершения преступления}
\begin{itemize}
  \item Фишинг
  \item Спам-рассылки
  \item Социальная инженерия
  \item Распространение нелицензионного ПО, содержащего вирусы
\end{itemize}

\subsection*{Мотив}
\begin{itemize}
  \item Деньги
  \item Самоудовлетворение
\end{itemize}

\subsection*{Круг подозреваемых}
Любой.

\subsection*{Применяемые статьи УК}
\begin{itemize}
  \item 282 УК РФ. Неправомерный доступ к компьютерной информации.
  \item УК РФ, Статья 163. Вымогательство.
\end{itemize}

\subsection*{Лица несущие ответственность}
Злоумышленник / группа злоумышленников.

\subsection*{Литература} УК РФ.
