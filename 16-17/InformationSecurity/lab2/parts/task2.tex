\section*{Сценарий 2}

\subsection*{Задание}
Злоумышленники совершили ряд онлайн покупок с использованием банковской карты
жертвы, при условии, что карта находится у владельца.

\subsection*{Тип преступления}
\begin{itemize}
  \item Мошенничество.
  \item Кража компьютерной информации.
\end{itemize}

\subsection*{Способ совершения преступления}
\begin{itemize}
  \item Мошенничество.
  \item Кража личности.
  \item Социальная инженерия.
  \item Распространение нелицензионного ПО, содержащего вирусы.
\end{itemize}

\subsection*{Мотив}
\begin{itemize}
  \item Деньги.
  \item Самоудовлетворение.
  \item Самосовершенствование и мастерство.
\end{itemize}

\subsection*{Круг подозреваемых}
Любой.

\subsection*{Применяемые статьи УК}
\begin{itemize}
  \item 282 УК РФ. Неправомерный доступ к компьютерной информации.
  \item УК РФ, Статья 159.3. Мошенничество с использованием платежных карт.
  \item УК РФ, Статья 158. Кража.
\end{itemize}

\subsection*{Лица несущие ответственность}
Злоумышленник / группа злоумышленников.

\subsection*{Литература} УК РФ.
