\section{Вопросы}

\subsection{Что дает присвоение каждому пользователю
  уникального идентификатора?}

  Это позволяет вводить различный уровень прав доступа пользователей к
  информации и позволяет возможность полного учета всех входов пользователей в
  систему в журнале аудита.

\subsection{Чем определяется стойкость к взлому подсистемы
  идентификации и аутентификации?}

  Видом идентификатора, способом аутентификации и тем как организован обмен
  данными для аутентификации.

\subsection{Чем определяется сложность подбора пароля? Как
  производится количественная оценка стойкости парольных
  систем?}

  Количественная оценка может быть выполнена по формуле:
  \begin{equation}
    P = \frac{V\times T}{S} = \frac{V\times T}{A^L},
  \end{equation}
  где $S=A^L$ --- число возможных паролей длины $L$, которые можно составить из
  алфавита мощности $A$; $V$ --- скорость перебора паролей; $T$ ---
  максимальный срок действия паролей.

\subsection{Сравните сложность подбора представленных паролей:
  \textit{18\char`_JcT*a} (символы верхнего и нижнего регистров, цифры,
  специальные символы) и \textit{Jf1UGwxRd} (символы верхнего и нижнего
  регистров, цифры).}

  Положим скорость перебора паролей разной, тогда разницу в формулу включает
  только мощность алфавита и длина пароля, причем длина пароля является
  показателем степени и вносит больший вклад. Тогда сложность перебора пароля
  длины 9 выше не смотря на менее мощный алфавит.

\subsection{Какие недостатки есть у такого метода
  противодействия подбору паролей, как ограничение числа
  попыток ввода пароля? Чем его можно заменить?}

  Недостаток в том, что вероятность подбора пароля за ограниченное число
  попыток все же существует, а так же существует вероятность частой блокировки
  ''забывчивых'' пользователей. В качестве решения проблемы можно использовать
  трехфакторную аутентификацию.

\subsection{Как изменится стойкость к взлому подсистемы
  парольной аутентификации при увеличении характеристик A, L,
  V, T? При их уменьшении?}

  Изменение параметров $V$ и $T$ вносят линейный вклад в формулу расчета
  стойкости (при уменьшении параметров увеличивается стойкость). Изменение
  параметров $A$ и $L$ изменяют стойкость по степенному и показательному закону
  соответственно (при увеличении параметров увеличивается стойкость).

\subsection{В каком виде пароли могут храниться в БД учетных
  записей? Опишите недостатки этих видов хранения.}

  Пароли в базе данных можно хранить в шифрованном или хешированном виде.

  Для шифрованного хранения необходимо обеспечить безопасный обмен ключом
  шифрования. В случае получения злоумышленником ключа шифрования у него будет
  возможность получить пароли пользователей в открытом виде.

  Недостатком хеширования является невозможность восстановления пароля
  пользователя.

\subsection{Какой метод может применяться для сокрытия паролей
  в БД от администратора. Как этот метод может быть усилен для
  предотвращения подбора паролей?}

  Хеширование паролей на клиентской стороне. Автоматическая генерация паролей,
  принудительная смена ''пароля по умолчанию''.

\subsection{Приведите примеры технических устройств, с помощью
  которых может решаться задача идентификации и аутентификации
  пользователя?}

  \begin{itemize}
    \item USB ключи
    \item Пластиковые карты
    \item Идентификаторы iButton
    \item Бесконтактные радиочастотные карты proximity
  \end{itemize}

\subsection{Какие биометрические характеристики применяются для
  аутентификации? В чем преимущества этого способа 
  аутентификации?}

  \begin{itemize}
    \item Отпечатки пальцев
    \item Геометрическая форма рук
    \item Особенности голоса
    \item Рисунок радужной оболочки глаза
    \item Форма и размеры лица
  \end{itemize}

  Преимуществом является то, что зачастую очень сложно подделать биометрические
  параметры, и для аутентификации необходимо иметь часть тела пользователя.

