\begin{problem}
  Шифровка дихеши с использование военного шифра второй половины XIX века.

  \begin{enumerate}
    \item Зашифровать депешу: ''\textit{Милостивый государь! Войны начинаются в
      умах людей. Из преамбулы Устава ЮНЕСКО. Целью войны является мир.
      Аристотель.  Москва. С истинным почтением имею честь быть!}''. При
      шифровании необходимо реализовать все возможности данного алгоритма
      шифрования, повышающие его криптостойкость.

    \item Укажите хотя бы три недостатка используемых шифрообозначений в
      данном конкретном шифре.
  \end{enumerate}
\end{problem}

\begin{solution}
  Для начала исключим из исходного текста частоупотребимые стандартные фразы:
  \begin{addmargin}[2cm]{1cm}
    \textit{Войны начинаются в умах людей. Из преамбулы Устава ЮНЕСКО. Целью
    войны является мир. Аристотель.}
  \end{addmargin}

  Зашифруем исходный текст по словарю:
  \begin{addmargin}[2cm]{1cm}
    152 640 364 565 364 693 457 140 485 100 520 320 690 185 300 283 400 424 109
    139 320 640 494 100 690 369 440 307 554 690 152 640 695 140 329 213 457 351
    100 429 440 472 463 320 660
  \end{addmargin}

  Вставим в шифр несколько пустышек:
  \begin{addmargin}[2cm]{1cm}
    152 640 364 752 565 364 693 457 140 831 485 100 520 831 320 690 185 300 900
    283 400 424 109 139 320 640 494 845 100 690 369 440 777 307 554 690 152 640
    983 695 140 329 213 457 900 351 100 429 900 440 472 765 463 320 799 660
  \end{addmargin}

  Вставим в шифр несколько намеренных ошибок и групп-уточнителей после них
  пустышек:
  \begin{addmargin}[2cm]{1cm}
    152 640 364 752 565 364 693 457 140 831 485 100 520 831 320 690 185 300 900
    283 400 454 675 239 631 424 109 139 320 640 494 845 100 690 369 440 777 307
    554 690 152 640 983 695 140 329 213 457 900 351 100 420 675 239 631 429 900
    440 472 765 463 320 799 660
  \end{addmargin}

  Недостатки шифра:
  \begin{enumerate}
    \item Небольшая частотность пустышек в тексте, из-за чего их можно
      установить.
    \item Шифр возможно разгадать с помощью частотного анализа.
    \item При шифровании большого числа сообщений таким словарем снижается
      криптоустойчивость (легче применить частотный анализ).
  \end{enumerate}
\end{solution}
