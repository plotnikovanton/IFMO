\section*{Сценарий 7}

\subsection*{Задание}
Злоумышленники занимаются распространением аудио-контента с использованием
электронных ресурсов (сайты, соц. сети, форумы, торренты)

\subsection*{Тип преступления}
\begin{itemize}
  \item Незаконный доступ к охраняемой информации.
  \item Незаконное использование объектов авторского права или смежных прав с
    целью сбыта.
  \item Пиратство, распространение ПО и мультимедийного контента.
\end{itemize}

\subsection*{Способ совершения преступления}
\begin{itemize}
  \item Распространение нелицензионного аудио-контента на электронных ресурсах.
\end{itemize}

\subsection*{Мотив}
\begin{itemize}
  \item Самоудовлетворение.
\end{itemize}

\subsection*{Круг подозреваемых}
Лица, размещающие контент. 

\subsection*{Применяемые статьи УК}
\begin{itemize}
  \item 272 УК РФ. Неправомерный доступ к компьютерной информации.
  \item УК РФ, Статья 146. Нарушение авторских и смежных прав
  \item КоАП РФ, Статья 7.12. Нарушение авторских и смежных прав,
    изобретательских и патентных прав
\end{itemize}

\subsection*{Лица несущие ответственность}
Злоумышленник / группа злоумышленников.

\subsection*{Литература} КоАП. УК РФ.
