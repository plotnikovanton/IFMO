\section{Заключение}

Нами были рассмотрены понятия аналогии в целом и аналогии в математике.

На примерах конкретных задач была рассмотрена полезность метода аналогий,
интуитивно возникающих у математиков, при решении тех или инх задач.

Также можно сделать вывод о том, что аналогии в математическом мышлении
являются одним из самых важных инструментов. Именно видение аналогий позволяет
делать многие интуитивные предположения при решении зачач или доказательствах
теорем.

Таким образом математическая аналогия:

\begin{enumerate}
  \item позволяет получать новые математические результаты; устанавливает связи
    между различными областями математики, способствуя образованию единой
    системы знаний.

  \item Аналогия имеет большое значение при обучении математике, так как она
    позволяет добиться более глубокого понимания материала и помогает решать
    конкретные математические задачи.
\end{enumerate}
