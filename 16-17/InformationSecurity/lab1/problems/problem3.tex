\begin{problem}
  Зашифровать депешу с использованием агентурного шифра системы ''шифр
  Цезаря''.

  Исходный текст: влияние рассеянного немонохроматичного излyчения низкой
  интенсивности на yглеродистые стали (Воздействие лyнного света на рельсы).
\end{problem}

\begin{solution}
  Решение получено автоматически:
  \begin{minted}
  [
  frame=lines,
  framesep=2mm,
  baselinestretch=1.2,
  fontsize=\footnotesize,
  linenos
  ] 
  {python}
#!/bin/python3

d = {
    'а': '60',
    'б': '73',
    'в': '85',
    'г': '98',
    'д': '11',
    'е': '24',
    'ё': '37',
    'ж': '50',
    'з': '63',
    'и': '76',
    'й': '89',
    'к': '02',
    'л': '15',
    'м': '28',
    'н': '41',
    'о': '54',
    'п': '67',
    'р': '80',
    'с': '93',
    'т': '06',
    'у': '19',
    'ф': '32',
    'х': '45',
    'ц': '58',
    'ч': '71',
    'ш': '84',
    'щ': '97',
    'ъ': '10',
    'ы': '23',
    'ь': '36',
    'э': '49',
    'ю': '62',
    'я': '75'
}

for c in input().lower():
    if c in d:
        print(d[c], end=' ')
  \end{minted}


  Зашифрованный текст:

  \begin{addmargin}[2cm]{1cm}
    юзеыйеб мьннбыййкяк йбикйксмкиьоеуйкяк едзубйеы йеджкё ейобйнеюйкное йь
    язбмкаеночб ноьзе юкдабёноюеб зййкяк нюбоь йь мбзшнч
  \end{addmargin}

\end{solution}
